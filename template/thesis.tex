%\documentclass[german,bachelor]{swsLeipzig}
\documentclass[german,bachelor]{swsLeipzig}
% When you change the language, pdflatex may halt on recompilation.
% Just hit enter to continue and recompile again. This should fix it.

%
% Values
% ------
\ThesisSetTitle{Ermittlung von Konfigurationsoptionen im Source Code mit Fokus auf Machine Learning Bibliotheken in Python}
\ThesisSetKeywords{These, Are, My keywords} % only for PDF meta attributes

\ThesisSetAuthor{Marco Jaeger-Kufel}
\ThesisSetStudentNumber{3731679}
\ThesisSetDateOfBirth{10}{05}{1995}
\ThesisSetPlaceOfBirth{Hannover}

\ThesisSetSupervisors{Prof.\ Dr.\ Norbert Siegmund,Prof.\ Dr.\ Unknown Yet}

\ThesisSetSubmissionDate{14}{06}{2022}

%
% Suggested Packages
% ------------------
\usepackage[sort&compress]{natbib}
%   Allows citing in different ways (e.g., only the authors if you use the
%   citation again within a short time).
%
\usepackage{booktabs}
%    For tables ``looking the right way''.
%
%\usepackage{tabularx}
%    Enables tables with columns that automatically fill the page width.
%
%\usepackage[ruled,algochapter]{algorithm2e}
%    A package for pseudo code algorithms.
%
%\usepackage{amsmath}
%    For tabular-style formatting of mathematical environments.
%

%
% Commenting (by your supervisor)
% -------------------------------
\usepackage{xcolor}
\usepackage{soul}
\usepackage{listings}
\newcommand{\bscom}[2]{%
  % #1 Original text.
  % #2 Replacement text.
    \st{\scriptsize\,#1}{\color{blue}\scriptsize\,#2}%
  }

% Create links in the pdf document
% Hyperref has some incompatibilities with other packages
% Some other packages must be loaded before, some after hyperref
% Additional options to the hyperref package can be provided in the braces []
\usehyperref[backref] % This will add back references in the bibliography
\usepackage[utf8]{inputenc}
\usepackage{textcomp}
\setcitestyle{authoryear,close={)}}

\begin{document}
\begin{frontmatter}
  \begin{abstract}
    A short summary.
  \end{abstract}

  \tableofcontents

  %\chapter*{Acknowledgements} % optional
  %I thank the authors of the swsLeipzig template for their excellent work!

  %\listoffigures % optional

  %\listoftables % optional

  %\listofalgorithms % optional
  %    requires package algorithm2e

  % optional: list of symbols/notation (e.g., using the nomencl package)
\end{frontmatter}

\chapter{Einleitung}\label{Einleitung}
Diese Arbeit entsteht am Institut für Informatik an der Universit\"at Leipzig in der Abteilung \glqq Softwaresysteme\grqq.
Im Folgenden wird die zugrundeliegende Problemstellung und Relevanz erl\"autert.
Anschlie\ss end wird dieses Problem auf einen konkreten Anwendungsbereich \"ubertragen und auf die Zielsetzung der Arbeit eingegangen.\\

\section{Motivation}
Moderne Softwaresysteme erm\"oglichen den Nutzenden ein breites Spektrum unterschiedlicher Konfigurationsoptionen.
Anhand dieser Konfigurationsoptionen sind die Nutzenden in der Lage, viele Aspekte der Ausgestaltung einer Software zu steuern.
Konfigurationsoptionen k\"onnen dabei ganz unterschiedliche Funktionen besitzen, die von den Nutzenden nach den eigenen Bed\"urfnissen angepasst werden k\"onnen.\\

Konfigurationsoptionen können in verschiedenen Teilen eines Softwareprojekts verarbeitet, definiert und beschrieben werden:
in der Konfigurationsdatei, im Source Code und in der Dokumentation \cite[S. 185]{7774519}.
Sie werden meist als Key-Value Pair entworfen und gesammelt in einer Konfigurationsdatei gespeichert.
Dem Namen der Konfigurationsoption (Key) werden dabei Einstellungsmöglichkeiten beliebigen Typs zugeordnet (Value).
Zur Speicherung von Konfigurationsoptionen verwenden einige Systeme, wie das Big Data-Framework Hadoop, strukturierte XML-Formate oder auch JSON-Dateien.
Ein einheitliches Schema zur Speicherung als Konfigurationsdatei gibt es jedoch nicht, weshalb sich diese von der Struktur und Syntax unterscheiden. \cite[S. 131]{10.1145/1985793.1985812}.
Im Source Code können die Konfigurationsoptionen eingelesen und bearbeitet werden \cite[S. 185]{7774519}.\\

W\"ahrend die Vielf\"altigkeit und Individualisierbarkeit der Software gr\"o\ss er wird, erh\"oht sich auch die Komplexit\"at der Software f\"ur Entwickelnde.
Vor allem das Warten und Testen gestaltet sich nun umfangreicher.
Besonders problematisch wird es, wenn Konfigurationsoptionen auf unvorhergesehene Weise interagieren.
Solche Abh\"angigkeiten sind in einem wachsenden Spielraum m\"oglicher Kombinationen schwer zu entdecken und zu verstehen.
Entwickelnde m\"ussen die Konfigurationsoptionen innerhalb der Software verfolgen, um festzustellen,
welche Codefragmente von einer Option betroffen sind und wo und wie sie mit ihr interagieren.
Des Weiteren kann es vorkommen, dass Entwickelnde Werte von Konfigurationsoptionen nicht aktualisieren, was dazu führen kann,
dass über mehrere Module hinweg, unbemerkt mit falschen Werten gearbeitet wird \cite[S. 185]{7774519}.
Die verschiedenen m\"oglichen Auspr\"agungen einer Konfigurationsoption erschweren hierbei die Nachvollziehbarkeit des Kontrollflusses der Software.\\

In einer empirischen Studie über Konfigurationsfehler stellen \citeauthor{10.1145/2043556.2043572} (\citeyear{10.1145/2043556.2043572}, S. 160) fest,
dass in kommerziellen Softwareunternehmen für Speicherlösungen, knapp ein Drittel aller Ursachen von Kundenproblemen auf Konfigurationsfehler zurückzuführen sind (siehe Tabelle 3).
Bei Konfigurationsfehler sind Source Code und die Eingabe zwar korrekt, es wird jedoch ein falscher Wert für eine Konfigurationsoption verwendet,
sodass sich die Software nicht wie gewünscht verhält \cite[S. 152]{10.1145/2568225.2568251}.
Solche Fehler können dazu führen, dass die Software abstürzt, eine fehlerhafte Ausgabe erzeugt oder nur unzureichend funktioniert \cite[S. 152]{10.1145/2568225.2568251}.\\

Konfigurationsfehler entstehen zum Beispiel durch die Verwendung unterschiedlicher Versionen einer Software.
Im folgendem Codeausschnitt wird die scikit-learn Klasse \textit{LogisticRegression} initialisiert und der Variable \textit{clf} zugewiesen:\\

\begin{lstlisting}[language=Python, frame=single]
  from sklearn.linear_model import LogisticRegression
clf = LogisticRegression()
\end{lstlisting}
\

Da keine Parameter angegeben werden, wird die Klasse mit den Default-Werten initialisiert, zum Beispiel
\textit{max\_iter = 100, verbose = 0, n\_jobs = 1}.
In der Version \textit{scikit-learn 0.21.3} wird für den Parameter \textit{multi\_class} als Default noch der Wert \textit{ovr} zugewiesen (\citeyear{sklearn}).
Für alle nachfolgenden Versionen ist der Default-Wert für diesen Parameter jedoch \textit{auto} (\citeyear{sklearn}).
Folglich führt die Verwendung dieser Klasse ohne explizite Parameterangabe, nach Verwendung unterschiedlicher Versionen, zu schwer nachvollziehbaren Programmverhalten.\\

\section{Anwendungsbereich und Zielsetzung}
Das Ziel dieser Arbeit ist es, Konfigurationsoptionen im Source Code zu erkennen und zu extrahieren. \\

Es existieren bereits einige Forschungsansätze zur Ermittlung und Verarbeitung von Konfigurationsoptionen im Source Code.
Viel zitiert wird dabei der Ansatz von \citeauthor{10.1145/1985793.1985812}, den sie \citeyear{10.1145/1985793.1985812} publizierten.
Wie auch \citeauthor{7774519} oder \citeauthor{8049300} entwickelten sie einen Ansatz, mittels statischer Code-Analyse Konfigurationsoptionen
im Source Code zu tracken.
Dabei fokussierten sie sich auf die Programmiersprache Java, die nach dem TIOBE-Index jahrelang als beliebteste Programmiersprache galt (\citeyear{enwiki:1077809155}).
Der TIOBE-Index misst die Popularität von Programmiersprachen auf der Grundlage von Suchanfragen auf beliebten Websites und in Suchmaschinen (\citeyear{enwiki:1077809155}).
Im Februar 2022 ist in diesem Ranking Python erstmals zur beliebtesten Programmiersprache aufgestiegen (\citeyear{enwiki:1077809155}).
Für Python ist diese Thematik jedoch bislang allerdings noch wenig beleuchtet. \\

Die Programmiersprache Python ist für ihre Benutzerfreundlichkeit bekannt und obwohl Python eine interpretierte High-Level-Programmiersprache ist,
ist sie in der Lage, bei Bedarf die Leistung von Programmiersprachen auf Systemebene zu nutzen \cite[S. 2]{2020}.
Insbesondere im Bereich wissenschaftliches Rechnen (Scientific Computing) gewann Python in den letzten Jahren enorm an Popularität,
weshalb viele Bibliotheken für maschinelles Lernen auf Python basieren \cite[S. 2]{2020}. \\

Gegenstand dieser Arbeit werden daher Konfigurationsoptionen sein, die in Python Source Code vorliegen und mittels statischer Code-Analyse erkannt werden.
Der Ansatz identifiziert automatisch die Stellen im Source Code, an denen die Optionen gelsesen und ermittelt für jede dieser Stellen den Namen,
sowie die übergebenen Werte der Option.
Dabei liegt der Fokus auf drei der populärsten Machine Learning Bibliotheken in dieser Sprache (siehe stat): TensorFlow, PyTorch und scikit-learn.
Zudem wird die Verwendung von Konfigurationsoptionen der Machine Learning Lifecylce Plattform Mlflow untersucht. \\


\section{Aufbau dieser Arbeit und methodisches Vorgehen}
% Umsetzung gleidert isch in 3 Schritte: WEbscrping, nochmal auf Klassen eingehen, Statische Code-Analyse mit ReadPoint, und extrahieren der Parameter,
% Am Ende einfpelgen in s Cg net

\chapter{Hintergrund}\label{Hintergrund}

\section{Statische Code-Analyse}

Das wichtigste Werkzeug dieser Arbeit ist die statische Code-Analyse,
mit der Software-Projekte unabhängig von der Ausführungsumgebung untersucht werden können.
Die statische Code-Analyse ist ein Werkzeug, um Fehler in einer Softwareanwendung zu reduzieren \cite[S. 99]{bardas2010static}.
So ermöglichen sie den Anwendenden, Fehler in einem Programm zu finden, die für den Compiler nicht sichtbar sind \cite[S. 99]{bardas2010static}.\\

Im Gegensatz zur dynamischen Analyse wird die statische Analyse zur Übersetzungszeit durchgeführt
und setzt damit bereits vor der tatsächlichen Ausführung des Source Codes an \cite[S. 2]{gomes2009overview}.
Die erzeugten Ergebnisse der statischen Analyse lassen sich besser Verallgemeinern, da sie nicht abhängig von den Eingaben sind,
mit denen das Programm während der dynamischen Analyse ausgeführt wurde \cite[S. 6]{gomes2009overview}.\\

Für das Aufspüren von Konfigurationsoptionen bietet sich eine statische Code-Analyse daher aus mehreren Gründen an.
So kann es viele Optionen geben, die nur in bestimmten Modulen oder als Folge bestimmter Eingaben verwendet werden.
Es ist unwahrscheinlich, dass mittels dynamischen Testens alle Verwendungen der gesuchten Konfigurationsoptionen gefunden werden.
Dies würde zudem bei größeren Softwareprojekten eine sehr komplexe Test-Suite erfordern, um möglichst alle Fälle abdecken zu können.
Mit der statischen Code-Analyse kann eine hohe Abdeckung hingegen deutlich leichter erreicht werden.
Gleichzeitig verbergen sich hinter den Methoden und Klassen, der zu untersuchenden Machine Learning Bibliotheken,
teils sehr komplexe und rechenintensive Berechnungen, die bis zu mehrere Tage laufen könnten.
Aus Kosten-Nutzen-Gründen ist hier eine dynamische Analyse nur bedingt sinnvoll.\\


\section{Machine Learning}
%Maschinelles Lernen (ML) wird eingesetzt, um Maschinen beizubringen, wie sie Daten effizienter verarbeiten können.
%Manchmal können wir nach der Sichtung der Daten die Informationen, die wir aus den Daten gewinnen, nicht interpretieren.
%In diesem Fall wenden wir maschinelles Lernen an.
%Mit der Fülle der verfügbaren Datensätze steigt auch die Nachfrage nach maschinellem Lernen.
%Viele Branchen setzen maschinelles Lernen ein, um relevante Daten zu extrahieren.
%Der Zweck des maschinellen Lernens besteht darin, aus den Daten zu lernen.
%Es wurden viele Studien darüber durchgeführt, wie Maschinen selbständig lernen können, ohne explizit programmiert zu werden.
%Viele Mathematiker und Programmierer wenden verschiedene Ansätze an, um eine Lösung für dieses Problem zu finden, das riesige Datensätze umfasst.

\subsection{Grundlagen}
Die künstliche Intelligenz (KI) ist ein Teilgebiet der Informatik und befasst sich mit der Entwicklung von Computerprogrammen und Maschinen,
die in der Lage sind, Aufgaben auszuführen, die Menschen von Natur aus gut beherrschen \cite[S. 1]{2020}.
Dazu gehören zum Beispiel die Verarbeitung von natürlicher Sprache (Natural Language Processing) oder Bilderkennung (Computer Vision).
In der Mitte des 20. Jahrhunderts entstand das maschinelle Lernen (ML) als Teilbereich der KI und schlug eine neue Richtung
für die Entwicklung von künstlicher Intelligenz ein, inspiriert vom konzeptionellen Verständnis der Funktionsweise des menschlichen Gehirns \cite[S. 1]{2020}.\\

Pionier Arthur Samuel definiert maschinelles Lernen als ein Fachgebiet, das Computern die Fähigkeit verleiht, zu lernen,
ohne ausdrücklich programmiert zu werden \cite[S. 381]{mahesh2020machine}.
Es befasst sich mit der wissenschaftlichen Untersuchung von Algorithmen und statistischen Modellen,
die Computersysteme verwenden, um eine Aufgabe zu lösen \cite[S. 381]{mahesh2020machine}.
Dabei werden statistische Methoden verwendet, um aus Daten zu lernen und Muster zu erkennen. \\

%% schreiben über zunehmende Anforderungen und wie wichtig, ML nun ist

Im Allgemeinen werden die Ansätze des maschinellen Lernens in drei große Kategorien unterteilt, je nach Art des \textit{Signals}
oder \textit{Feedbacks}, das dem lernenden System zur Verfügung steht \cite[S. 2]{cite-key}:
\begin{itemize}
 \item Überwachtes Lernen (supervised learning)
 \item Unüberwachtes Lernen (unsupervised learning)
 \item Bestärkendes Lernen (reinforcement learning)
\end{itemize}

Das \textit{überwachte Lernen} erfordert ein Training mit gelabelten Daten, die Eingaben und gewünschte Ausgaben haben \cite[S. 2]{cite-key}.
So weiß das Modell zum Beispiel, ob auf einem bestimmten Foto ein Hund abgebildet ist oder wie viel ein bestimmtes Haus kostet.
Auf dieser Grundlage kann es dann so trainiert werden, dass es neue Hunde erkennt und den Preis für neue Häuser schätzt. \\

Im Gegensatz dazu erfordert das \textit{unüberwachte Lernen} keine markierten Trainingsdaten und erhält lediglich Eingabedaten \cite[S. 2]{cite-key}.
Die richtige Antwort ist entweder nicht bekannt oder existiert nicht.
Stattdessen sucht das Modell nach sinnvollen Strukturen in den Daten, um neue Erkenntnisse über ein Thema zu gewinnen \cite[S. 383]{mahesh2020machine}.
Unüberwachtes Lernen kann zum Beispiel dazu verwendet werden, Kunden zu finden, die einen ähnlichen Geschmack haben,
und ihnen Artikel empfehlen, die diese Kunden gekauft haben. \\

Das \textit{bestärkende Lernen} hingegen unterscheidet sich sehr deutlich von den beiden vorherigen Kategorien.
Das Modell ist hier ein aktiver Akteur, der mit seiner externen Umgebung interagiert
und positives oder negatives Feedback für seine Handlungen erhält \cite[S. 2]{cite-key}.
Aus diesen Rückmeldungen erlernt es optimale Handlungsstrategien für seine Umgebung zu entwickeln \cite[S. 384]{mahesh2020machine}.
Solche Modelle können zum Beipiel für selbstfahrende Autos verwendet werden oder auch um einer Maschine Gesellschaftsspiele beizubringen.




\subsection{Python-Bibliotheken}
% Vorteile von Biblitoheken
% Grafik mit populärsten ML-Bibliotheken

\subsection{Konfigurationsoptionen}


\section{Web Scraping}


% Showing natbib citation commands
%Let us get started by citing \citet{manning:1999}!
%So what did \citeauthor{manning:1999} do in \citeyear{manning:1999}?
%Good question!
% Showing hyperref reference commands
%Maybe it is answered in \autoref{introduction} on page~\pageref{introduction}?
%Just to have something show up in the list of figures, I included \autoref{fig:a}.
%\begin{figure}[bt]% bottom or top of page (for small figures/tables)
 % \begin{center}{\huge\bf A}\end{center}
 % \caption{The first letter in the Roman alphabet.}\label{fig:a}
%\end{figure}
% \formatdate (or \formatdateshort)
%This date does not exist: \formatdateshort{30}{2}{2014}
%and is the same as \formatdate{30}{2}{2014}.
% An example table
%And here is some table with some numbers (\autoref{tab:numbers})
%which deserves to be on an extra page.
%\begin{table}[p]% extra page (usually for large figures/tables)
  %\caption{Tables have their captions above, figures below.}
  %\begin{center}
    %\begin{tabular}{lccc}\toprule
      %\multicolumn{4}{c}{Some numbers}\\\midrule
      %& 1999 & 2000 & 2001 \\\cmidrule(l){2-4}
      % cmidrule: A line from 2nd to 4th column, trimmed on the left hand side
      %Distance (km) & 23 & 18 & 42 \\
      %Awesomeness (aws) & 3.2 & 8.1 & 2.4 \\\bottomrule
    %\end{tabular}
  %\end{center}\label{tab:numbers}%
%\end{table}

% Appendix
\appendix
\chapter{My First Appendix}
This was just missing.

% Bibliography
\bibliographystyle{plainnat} % requires package natbib. An alternative is apalike
\bibliography{literature}    % load file literature.bib

\end{document}
