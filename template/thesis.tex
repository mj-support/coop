%\documentclass[german,bachelor]{swsLeipzig}
\documentclass[german,bachelor]{swsLeipzig}
% When you change the language, pdflatex may halt on recompilation.
% Just hit enter to continue and recompile again. This should fix it.

%
% Values
% ------
\ThesisSetTitle{Ermittlung von Konfigurationsoptionen im Source Code mit Fokus auf Machine Learning Bibliotheken in Python}
\ThesisSetKeywords{These, Are, My keywords} % only for PDF meta attributes

\ThesisSetAuthor{Marco Jaeger-Kufel}
\ThesisSetStudentNumber{3731679}
\ThesisSetDateOfBirth{10}{05}{1995}
\ThesisSetPlaceOfBirth{Hannover}

\ThesisSetSupervisors{Prof.\ Dr.\ Norbert Siegmund,Prof.\ Dr.\ Unknown Yet}

\ThesisSetSubmissionDate{14}{06}{2022}

%
% Suggested Packages
% ------------------
\usepackage[sort&compress]{natbib}
%   Allows citing in different ways (e.g., only the authors if you use the
%   citation again within a short time).
%
\usepackage{booktabs}
%    For tables ``looking the right way''.
%
%\usepackage{tabularx}
%    Enables tables with columns that automatically fill the page width.
%
%\usepackage[ruled,algochapter]{algorithm2e}
%    A package for pseudo code algorithms.
%
%\usepackage{amsmath}
%    For tabular-style formatting of mathematical environments.
%

%
% Commenting (by your supervisor)
% -------------------------------
\usepackage{xcolor}
\usepackage{soul}
\newcommand{\bscom}[2]{%
  % #1 Original text.
  % #2 Replacement text.
    \st{\scriptsize\,#1}{\color{blue}\scriptsize\,#2}%
  }

% Create links in the pdf document
% Hyperref has some incompatibilities with other packages
% Some other packages must be loaded before, some after hyperref
% Additional options to the hyperref package can be provided in the braces []
\usehyperref[backref] % This will add back references in the bibliography
\usepackage[utf8]{inputenc}
\setcitestyle{authoryear,close={)}}

\begin{document}
\begin{frontmatter}
  \begin{abstract}
    A short summary.
  \end{abstract}

  \tableofcontents

  %\chapter*{Acknowledgements} % optional
  %I thank the authors of the swsLeipzig template for their excellent work!

  %\listoffigures % optional

  %\listoftables % optional

  %\listofalgorithms % optional
  %    requires package algorithm2e

  % optional: list of symbols/notation (e.g., using the nomencl package)
\end{frontmatter}

\chapter{Einleitung}\label{Einleitung}
Diese Arbeit entsteht am Institut für Informatik an der Universit\"at Leipzig in der Abteilung \glqq Softwaresysteme\grqq.
Im Folgenden wird die zugrundeliegende Problemstellung und Relevanz erl\"autert.
Anschlie\ss end wird dieses Problem auf einen konkreten Anwendungsbereich \"ubertragen und auf die Zielsetzung der Arbeit eingegangen.\\

Moderne Softwaresysteme erm\"oglichen den Nutzenden ein breites Spektrum unterschiedlicher Konfigurationsoptionen.
Anhand dieser Konfigurationsoptionen sind die Nutzenden in der Lage, viele Aspekte der Ausgestaltung einer Software zu steuern.
Konfigurationsoptionen k\"onnen dabei ganz unterschiedliche Funktionen besitzen, die von den Nutzenden nach den eigenen Bed\"urfnissen angepasst werden k\"onnen.\\

W\"ahrend die Vielf\"altigkeit und Individualisierbarkeit der Software gr\"o\ss er wird, erh\"oht sich auch die Komplexit\"at der Software f\"ur Entwickelnde.
Vor allem das Warten und Testen gestaltet sich nun umfangreicher.
Besonders problematisch wird es, wenn Konfigurationsoptionen auf unvorhergesehene Weise interagieren.
Solche Abh\"angigkeiten sind in einem wachsenden Spielraum m\"oglicher Kombinationen schwer zu entdecken und zu verstehen.
Entwickelnde m\"ussen die Konfigurationsoptionen innerhalb der Software verfolgen, um festzustellen,
welche Codefragmente von einer Option betroffen sind und wo und wie sie mit ihr interagieren.
Die verschiedenen m\"oglichen Auspr\"agungen einer Konfigurationsoption erschweren hierbei die Nachvollziehbarkeit des Kontrollflusses der Software.\\

In einer empirischen Studie über Konfigurationsfehler stellten \citeauthor{10.1145/2043556.2043572} (\citeyear{10.1145/2043556.2043572}, S. 160)  fest,
dass in kommerziellen Softwareunternehmen für Speicherlösungen, knapp ein Drittel aller Ursachen von Kundenproblemen auf Konfigurationsfehler zurückzuführen sind (siehe Tabelle 3).
Bei Konfigurationsfehler sind Source Code und die Eingabe zwar korrekt, es wird jedoch ein falscher Wert für eine Konfigurationsoption verwendet,
sodass sich die Software nicht wie gewünscht verhält \cite[S. 152]{10.1145/2568225.2568251}. Solche Fehler können dazu führen, dass die Software abstürzt,
eine fehlerhafte Ausgabe erzeugt oder nur unzureichend funktioniert \cite[S. 152]{10.1145/2568225.2568251}.\\

Konfigurationsoptionen können in verschiedenen Teilen eines Softwareprojekts verarbeitet, definiert und beschrieben werden:
im Source Code, in der Konfigurationsdatei und in der Dokumentation \cite[S. 185]{7774519}.
Sie werden meist als Key-Value Pair entworfen und gesammelt in einer Konfigurationsdatei gespeichert.
Dem Namen der Konfigurationsoption (Key) werden dabei Einstellungsmöglichkeiten beliebigen Typs zugeordnet (Value).
Zur Speicherung verwenden einige Systeme, wie das Big Data-Framework Hadoop, strukturierte XML-Formate oder auch JSON-Dateien.
Ein einheitliches Schema zur Speicherung als Konfigurationsdatei gibt es jedoch nicht, weshalb sich diese von der Struktur und Syntax unterscheiden. \cite[S. 131]{10.1145/1985793.1985812}.
Über den Source Code einer Software können die Konfigurationsoptionen eingelesen und bearbeitet werden \cite[S. 185]{7774519}.

\section{Anwendungsbereich und Zielsetzung}

\section{Aufbau dieser Arbeit und methodisches Vorgehen}

% Showing natbib citation commands
Let us get started by citing \citet{manning:1999}!
So what did \citeauthor{manning:1999} do in \citeyear{manning:1999}?
Good question!
% Showing hyperref reference commands
Maybe it is answered in \autoref{introduction} on page~\pageref{introduction}?
Just to have something show up in the list of figures, I included \autoref{fig:a}.
\begin{figure}[bt]% bottom or top of page (for small figures/tables)
  \begin{center}{\huge\bf A}\end{center}
  \caption{The first letter in the Roman alphabet.}\label{fig:a}
\end{figure}
% \formatdate (or \formatdateshort)
This date does not exist: \formatdateshort{30}{2}{2014}
and is the same as \formatdate{30}{2}{2014}.
% An example table
And here is some table with some numbers (\autoref{tab:numbers})
which deserves to be on an extra page.
\begin{table}[p]% extra page (usually for large figures/tables)
  \caption{Tables have their captions above, figures below.}
  \begin{center}
    \begin{tabular}{lccc}\toprule
      \multicolumn{4}{c}{Some numbers}\\\midrule
      & 1999 & 2000 & 2001 \\\cmidrule(l){2-4}
      % cmidrule: A line from 2nd to 4th column, trimmed on the left hand side
      Distance (km) & 23 & 18 & 42 \\
      Awesomeness (aws) & 3.2 & 8.1 & 2.4 \\\bottomrule
    \end{tabular}
  \end{center}\label{tab:numbers}%
\end{table}

% Appendix
\appendix
\chapter{My First Appendix}
This was just missing.

% Bibliography
\bibliographystyle{plainnat} % requires package natbib. An alternative is apalike
\bibliography{literature}    % load file literature.bib

\end{document}
